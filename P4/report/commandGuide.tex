%-------- misc. ----------
“ ”       
“”
%---- custom commands ----
\QED
\BigO{foo}
$\rfrac{a}{b}$

%-------- math -----------
% bold character in math environment 
$A\bm{x}=\bm{b}$

% variations of set symbols like N, R, and Z 
$\mathbbm{N}$

% slanted fractions (looks nice in text)
$\sfrac{1}{2}$

% nicely formatted integrals
$\int_{0}^{\infty} \mathrm{e}^{-x}\,\mathrm{d}x$
% and
$\int\limits_{0}^{\infty} \mathrm{e}^{-x}\,\mathrm{d}x$

For text between equations in a math env, use $\intertext{}$
\begin{align*}
    E = m c^2 
    \intertext{and}
    c^2 = a^2 + b^2
\end{align*}

%-------- figures --------
check out subcaption vs subfig
% simple figure
\begin{figure}[H]
\begin{center}\includegraphics[scale=0.5]{Figures/filename}
\end{center}
\caption{figure text}
\label{fig:fig1}
\end{figure}

% subfigure
\begin{figure}[H]
\centering
\subfloat[]{{\includegraphics[scale=0.5]{Figures/filename1}}}
\qquad
\subfloat[]{{\includegraphics[scale=0.5]{Figures/filename2}}}
\caption{figure text}
\label{fig:fig1}
\end{figure}

%-------- tables ---------
% simple table 
\begin{table}[H]
\caption{Table text}
\centering
\begin{tabular}{|c|c|c|}
\hline
bla & bla & bla
\\
\hline 
bla & bla & bla
\\
\hline
\end{tabular}
\label{tab:tab1}
\end{table}

%----------------- TODO -------------------
% beside text
\todo{Is it possible to add a subsubparagraph?}
% inline of text
\todo[inline,color=green]{I think that a summary of this exciting chapter should be added.}

%-------------- framed environment ---------------
% mdframed
\begin{mdframed}[style=MyFrame]
    $E = m c^2$
\end{mdframed}

% tcolorbox
% default: red frame
\begin{equation*}
\tcbhighmath[drop fuzzy shadow=blue]
{
    \begin{aligned}
        E = m c^2
    \end{aligned}
}
\end{equation*} 

% with blue frame
\begin{equation*}
\tcbhighmath[colframe=blue,drop fuzzy shadow=black]
{
    \begin{aligned}
        E = m c^2
    \end{aligned}
}
\end{equation*}

%------------------- code ------------------------
%---- lstlisting -----
\begin{lstlisting}[language=python]
print("Hello, World!)
\end{lstlisting}

%------ minted --------
\begin{minted}
[
frame=lines, % possible values are lines, leftline, topline, bottomlines and single
framesep=2mm,
baselinestretch=1.2,
bgcolor=LightGray,    
fontsize=\footnotesize,
linenos,    % enables line numbers
mathscape,  % enables math mode in code comments
rulecolor=LightGray,  % changes the colour of the frame
showspaces, % enables a special character to make spaces visible
]
{python}
print("Hello, World!")
\end{minted}

% including code from a file
\inputminted{cpp}{main.cpp}
% import only part of the file
\inputminted[firstline=2, lastline=12]{cpp}{main.cpp}
% one-line code
\mint{html}|<h2>Something <b>here</b></h2>|

% captions, labels and the list of listings
\begin{listing}[H]
\inputminted{cpp}{main.cpp}
\caption{Example from external file}
\label{listing:1}
\end{listing}
% to print the list with all listing elements use \listoflistings
\renewcommand\listoflistingscaption{List of source codes}
\listoflistings

%------------- cross-referencing -----------------
You can redefine the prefix for the autoref command. For instance for the sections use
\def\sectionautorefname{New prefix for the sections}
More generally
\def\<type>autorefname{<new name>}
Followed by
\addto\extrasenglish{%
  \renewcommand{\sectionautorefname}{Section}%
}

%href problem:
%\section{blabla $x$}
%gives error msg:
%token not allowed in a pdf string
%solution:
%\section{blabla \texorpdfstring{$x$}{Lg}}

%------------------- citation --------------------
\cite{MHJ}         %bare
\parencite{MHJ}    %cite in parentheses
\footcite{MHJ}     %cite in footnote
\authorcite{MHJ}   %cite only author
\titlecite{MHJ}    %cite only title
\yearcite{MHJ}     %cite only year
\urlcite{MHJ}      %cite only url

%------------------ algorithm ---------------------

\alglanguage{pseudocode}  % or \alglanguage{pascal}
\begin{algorithm}[H]
\caption{<your caption for this algorithm>}
\label{<your label for references later in your document>}
\begin{algorithmic}[1]  %[1] no. each line, [5] each 5th etc
<algorithmic environment>
\end{algorithmic}
\end{algorithm}

%------------ list of colors (xcolor) -------------
% basic
black
blue 
brown
cyan
darkgray
gray
green
lightgray
lime
magenta
olive
orange
pink
purple
red
teal
violet
white
yellow

% dvips
Apricot
Aquamarine
Bittersweet
BlueGreen
BlueViolet
BrickRed
BurntOrange
CadetBlue
CarnationPink
Cerulean
CornflowerBlue
Dandelion
DarkOrchid
Emerald
ForestGreen
Fuchsia
Goldenrod
GreenYellow
JungleGreen
Lavender
LimeGreen
Magenta
Mahogany
Maroon
Melon
MidnightBlue
Mulberry
NavyBlue
OliveGreen
OrangeRed
Orchid
Peach
Periwinkle
PineGreen
Plum
ProcessBlue
RawSienna
RedOrange
RedViolet
Rhodamine
RoyalBlue
RoyalPurple
RubineRed
Salmon
SeaGreen
Sepia
SkyBlue
SpringGreen
Tan
TealBlue	 	 	 
Thistle
Turquoise
VioletRed
WildStrawberry
YellowGreen
YellowOrange
